\section{Iterative Data Flow Analysis Framework}

\subsection{Data Flow Framework}

We pass $6$ key parameters to the data flow framework. These include the domain
which is represented as a vector of pointers, wherein each pointer refers to
an element in the domain.
Then, we pass the direction (forwards or backwards) and meet operator.
We define an abstract \texttt{transfer function}, which needs to be overridden
by each pass.\\

\texttt{virtual TransferOutput transferFn(BitVector input,
  std::vector<void*> domain,\\
  std::map<void*, int> domainToIndex,
  BasicBlock* block) = 0;
}\\

We only assume that this maps from an input bit vector to an output bit vector.
Finally, we define the boundary conditions for either the entry or exit basic
block, depending on the direction of the data flow analysis, and
all the inner basic blocks in the CFG.\\

\texttt{DataFlowResult DataFlow::run(Function \&F, std::vector<void*> domain,
  BitVector boundaryCond, BitVector initCond)}\\

Given a function \texttt{F}, we first identify the boundary block(s) and initialize them.
Then, we initialize the inner blocks and compute the neighbour list for each
basic block. Depending on the direction of the analysis, this is either the
set of successors or predecessors of the block in the CFG. We then pick the traversal order,
again based on the direction of the analysis.

Finally, we start processing the basic blocks along the traversal order
until we reach a fixed-point in our analysis.
In each iteration, we first apply the meet operator and then use
the transfer function. We return the \texttt{IN} and \texttt{OUT} sets for all basic blocks back to the
pass. Given this information, the pass can interpret the information as needed.

\subsection{Source Code Listing}

The listing starts from the next page.

\includepdf[pages=-]{../ClassicalDataflow/dataflow.pdf}

\section{Liveness Analysis}

\subsection{Implementation}

In this analysis, we compute the liveness of variables at each program point.
The domain is therefore the set of variables.
This is a backwards pass and the meet operator is UNION.
We initialize the exit block and all the interior blocks to $\phi$.
In each iteration, for each basic block other than the exit block, we
first compute the the \texttt{IN} and \texttt{OUT} sets thus :

\begin{align*}
\texttt{OUT}[\mathtt{B}] \; = \; \cup_{\mathtt{s} \in \texttt{successors}(\mathtt{B})} \texttt{IN}[\texttt{s}] \\
\texttt{IN}[\mathtt{B}] \; = \; \texttt{use}_{B} \cup (\texttt{OUT}[\mathtt{B}] - \texttt{def}_{\texttt{B}})
\end{align*}

Here,
$\texttt{def}_{B}$ is the set of variables that are defined in the block B, prior to any use of that
variable in the block B. $\texttt{use}_{B}$ is the set of variables whose values are used in B prior to any
definition of the variable in the block.
We handle $\phi$ instructions differently. We only compute $\texttt{def}_{B}$ but don't display any
information before these pseudo-instructions. While computing the sets during data flow analysis,
we propagate the use information for phi instructions onto the predecessor blocks. After we converge,
this pass shows the liveness of variables at each program point.

\subsection{Source Code Listing}

The listing starts from the next page.

\includepdf[pages=-]{../ClassicalDataflow/liveness.pdf}

\subsection{Test Cases}

The test cases start from the next page.

\includepdf[pages=-]{../ClassicalDataflow/test-inputs/complex.pdf}

\subsection{Expected Results}

\begingroup
\fontsize{6pt}{8pt}\selectfont
\begin{verbatim}
//===--------------------------------------------------------------------------------------------------------------------------===//
                                                   { %argc }
                                                   % cmp = icmp slt i32 %argc, 2
                                             { %argc, %cmp }
                                                              br i1 %cmp, label %if.then, label %lor.lhs.false
                                                   { %argc }
//===--------------------------------------------------------------------------------------------------------------------------===//
//===--------------------------------------------------------------------------------------------------------------------------===//
                                                   { %argc }
                                                   % cmp1 = icmp sle i32 undef, 0
                                            { %argc, %cmp1 }
                                                              br i1 %cmp1, label %if.then, label %if.end
                                                   { %argc }
//===--------------------------------------------------------------------------------------------------------------------------===//
//===--------------------------------------------------------------------------------------------------------------------------===//
                                                         { }
                                                              br label %return
                                                         { }
//===--------------------------------------------------------------------------------------------------------------------------===//
//===--------------------------------------------------------------------------------------------------------------------------===//
                                                   { %argc }
                                                   % add = add nsw i32 %argc, 1
                                             { %argc, %add }
                                             % mul = mul nsw i32 %argc, 5
                                             { %argc, %add }
                                             % div = sdiv i32 %add, 2
                                       { %argc, %add, %div }
                                       % cmp2 = icmp eq i32 %div, %argc
                                      { %argc, %add, %cmp2 }
                                                              br i1 %cmp2, label %if.then3, label %if.else
                                             { %argc, %add }
//===--------------------------------------------------------------------------------------------------------------------------===//
//===--------------------------------------------------------------------------------------------------------------------------===//
                                                   { %argc }
                                                   % rem = srem i32 %argc, 3
                                                   { %argc }
                                                   % rem4 = srem i32 %argc, 3
                                            { %argc, %rem4 }
                                            % rem5 = srem i32 %rem4, 3
                                     { %argc, %rem4, %rem5 }
                                     % mul6 = mul nsw i32 %rem4, 2
                                     { %argc, %rem4, %rem5 }
                                     % div7 = sdiv i32 %rem5, 5
                                     { %argc, %rem4, %div7 }
                                                              br label %if.end10
                                     { %argc, %rem4, %div7 }
//===--------------------------------------------------------------------------------------------------------------------------===//
//===--------------------------------------------------------------------------------------------------------------------------===//
                                             { %argc, %add }
                                             % div8 = sdiv i32 %add, 5
                                                   { %argc }
                                                   % rem9 = srem i32 %argc, 3
                                            { %argc, %rem9 }
                                                              br label %if.end10
                                            { %argc, %rem9 }
//===--------------------------------------------------------------------------------------------------------------------------===//
//===--------------------------------------------------------------------------------------------------------------------------===//
% n.0 = phi i32 [ %rem4, %if.then3 ], [ %argc, %if.else ]
% m.0 = phi i32 [ %div7, %if.then3 ], [ %rem9, %if.else ]
                                       { %argc, %n.0, %m.0 }
                                       % mul11 = mul nsw i32 %n.0, 2
                                       { %argc, %n.0, %m.0 }
                                       % div12 = sdiv i32 %m.0, 5
                                       { %argc, %n.0, %m.0 }
                                       % rem13 = srem i32 %n.0, 3
                                             { %argc, %m.0 }
                                             % mul14 = mul nsw i32 %m.0, 2
                                                   { %argc }
                                                              br label %return
                                                   { %argc }
//===--------------------------------------------------------------------------------------------------------------------------===//
//===--------------------------------------------------------------------------------------------------------------------------===//
% retval.0 = phi i32 [ -1, %if.then ], [ %argc, %if.end10 ]
                                               { %retval.0 }
                                                              ret i32 %retval.0
                                                         { }
//===--------------------------------------------------------------------------------------------------------------------------===//
\end{verbatim}
\endgroup

\section{Available Expressions Analysis}

\subsection{Implementation}

In this analysis, we compute the availability of expressions at each program point.
The domain is therefore the set of expressions.
This is a forwards pass and the meet operator is $\cap$.
We initialize the entry block to $\phi$  and all the interior blocks to $\mathtt{U}$.
In each iteration, for each basic block other than the entry block, we
first compute the the \texttt{IN} and \texttt{OUT} sets thus :

\begin{align*}
%\texttt{IN}[\mathtt{B}] \; = \; \cap_{\mathtt{p} \in \texttt{predecessors}(\mathtt{B})} \texttt{OUT}[\texttt{p}] \\
%\texttt{OUT}[\mathtt{B}] \; = \; \texttt{exp_gen}_{B} \cup (\texttt{IN}[\mathtt{B}] - \texttt{exp_kill}_{\texttt{B}})
\end{align*}

Here,
$\texttt{exp_gen}_{B}$ is the set of expressions that are generated in the block B, i.e. it computes these
expressions and does not subsequently redefine one of the operands in the expressions.
$\texttt{exp_kill}_{B}$ is the set of expressions that are killed in the block B, i.e. it
redefines one of the operands in the expressions and does not subsequently recompute that expression.

The interesting thing is that because we are working on code in SSA form, the computed values are never
destroyed. Thus, this analysis is very simple to reason about in this case.
We handle $\phi$ instructions differently. We only compute $\texttt{exp_kill}_{B}$ but don't display any
information before these pseudo-instructions.
After we converge, this pass shows the availability of expressions at each program point.

\subsection{Source Code Listing}

The listing starts from the next page.

\includepdf[pages=-]{../ClassicalDataflow/available.pdf}

\subsection{Test Cases}

The test cases start from the next page.

\includepdf[pages=-]{../ClassicalDataflow/test-inputs/complex.pdf}

\subsection{Expected Results}

\begingroup
\fontsize{6pt}{8pt}\selectfont
\begin{verbatim}
abc
\end{verbatim}
\endgroup

%%% Local Variables:
%%% mode: latex
%%% TeX-master: "asst1"
%%% End:
