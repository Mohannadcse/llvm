\section{FunctionInfo}

\subsection{Implementation}

We iterate over all functions in the module and for each function, we use helper functions in the
LLVM \texttt{Function} class, such as \texttt{getName()}, to obtain the required info.
We use the \texttt{size()} function in the \texttt{Instruction} class to compute the number of
instructions in all the basic blocks in the function.

\subsection{Source Code Listing}

The listing starts from the next page.

\includepdf[pages=-]{../FunctionInfo/FunctionInfo.pdf}

\subsection{Test Cases}

The test cases start from the next page.

\includepdf[pages=-]{../FunctionInfo/test-inputs/others.pdf}

\subsection{Expected Results}

\begin{verbatim}
Module test-inputs/others.bc
Name,   Args,   Calls,  Blocks, Insns
z,      0,      0,      1,      1
y,      1,      1,      3,      6
loop,   3,      1,      0,      0
FindMax,        *,      0,      8,      29
llvm.va_start,  1,      1,      0,      0
llvm.va_end,    1,      1,      0,      0
\end{verbatim}

\section{LocalOpts}

\subsection{Implementation}

We wrote a per-basic block pass that performs the required transformations. For each block, we keep
applying the 3 transformations within a \textbf{loop}, until the block is not modified after an entire iteration.
This is to ensure that the benefits of each optimization are relayed onto other optimizations and we
are not constrained by the order in which we apply the optimizations within a single iteration.
The transformations are described below:

\begin{itemize}
\item{\textbf{Algebraic Identities :}}
  We iterate over all instructions and look for binary operations involving scalar integer type operands with
  one of them being a constant.
  Depending on the type of operation, we apply identities that involve either the operator's \textbf{identity element} or
  \textbf{inverse element}. For instance, for the addition operator \texttt{Instruction::Add}, we use the identities
  $x + 0 = x$ and $0 + x = x$. For the subtraction operator \texttt{Instruction::Sub}, we apply the identities
  $x - 0 = x$ and $x - x = 0$. We do similar optimizations for other operators like  \texttt{Instruction::Mul},
  \texttt{Instruction::UDiv}, and  \texttt{Instruction::And}. In all cases, we use \texttt{ReplaceInstWithValue}
  to perform the optimization.
  
\item{\textbf{Constant Folding :}}
  We iterate over all instructions and look for binary operations involving scalar constant integer type operands.
  Depending on the type of operation, we evaluate the expression and replace all uses of that instruction with the
  constant using \texttt{ReplaceInstWithValue}. We do not handle floating point operations exhaustively.
  All common instructions like \texttt{Instruction::Add}, \texttt{Instruction::Sub}, \texttt{Instruction::Mul}, and
  \texttt{Instruction::Div} are handled.
  
\item{\textbf{Strength Reduction :}}
  We iterate over all instructions and look for binary operations involving scalar integer type operands with
  one of them being a constant.
  Depending on the type of operation, we perform strength reduction. For instance, when the RHS involves a
  \texttt{Instruction::Mul} with a number that is a power of 2, we replace that instruction with another instruction
  that uses \texttt{Instruction::Shl}. This is done using \texttt{ReplaceInstWithInst}. We also handle
  division by a power of 2. We extended the optimization to handle multiplication by numbers of the form $2^{k} + 1$
  and $2^{k} - 1$. For instance, multiplication with 3 is replaced by left shift and addition operators.  
  
\end{itemize}

\subsection{Source Code Listing}

The listing starts from the next page.

\includepdf[pages=-]{../LocalOpts/LocalOpts.pdf}

\subsection{Test Cases}

The test cases start from the next page.

\includepdf[pages=-]{../LocalOpts/test-inputs/merge.pdf}
\includepdf[pages=-]{../LocalOpts/test-inputs/others.pdf}

\subsection{Expected Results}

First test case :
\begin{verbatim}
Transformations applied:
Algebraic identities: 7
Constant folding: 19
Strength reduction: 0
\end{verbatim}
Second test case :
\begin{verbatim}
Transformations applied:
Algebraic identities: 0
Constant folding: 0
Strength reduction: 6
\end{verbatim}


%%% Local Variables:
%%% mode: latex
%%% TeX-master: "asst1"
%%% End:
