\section{LICM: Loop Invariant Code Motion}

\begin{itemize}

\item{
    \textbf{Loop-invariant instructions}\\

    The loop-invariant instructions are the following: $\{ S_{2}, S_{3}, S_{5}, S_{6}, S_{8}, S_{9}, S_{11}, S_{12} \}$.
    This is because the values of these computations  do not change as long as the control stays within the  loop.

    %These instructions are not loop-invariant : $\{ S_{1}, S_{4}, S_{7}, S_{10} \}$. They have multiple reaching definitions.
  }

  \item{
    \textbf{Loop-invariant instructions that can be moved to the pre-header}\\

    Only two instructions can be moved to the pre-header : $\{ S_{3}, S_{12} \}$. Both instructions satisfy all the three conditions required to be moved to the pre-header. In particular, $S_3$ can be moved because, while $q$ is live at the loop exit, $S_3$ dominates the loop exit. In contrast, $S_{12}$ does not dominate the loop exit, but $r$ is also not live at the loop exit. Further, when we move the two instructions to the pre-header, we must preserve their order: $S_{12}$ must be placed after $S_3$.

    The other instructions do not satisfy at least one of the three required conditions. For example, $S_5$, $S_6$, $S_9$, and $S_{11}$ cannot be moved because the variables defined by these instructions are live at the loop exit, and these instructions do not dominate the loop exit. $S_2$ and $S_8$ cannot be moved because the variables they define ($y$ and $g$ respectively) have other definitions inside the loop.
    
  }

\end{itemize}
