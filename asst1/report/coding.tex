\section{FunctionInfo}

\subsection{Implementation}

We iterate over all functions in the module and for each function, we use helper functions in the
LLVM \texttt{Function} class such as \texttt{getName()} to obtain the required info.
We use the \texttt{size()} function in the \texttt{Instruction} class to compute the number of
instructions in all the basic blocks in the function.

\subsection{Source Code Listing}

The listing starts from the next page.

\includepdf[pages=-]{../FunctionInfo/FunctionInfo.pdf}

\subsection{Test Cases}

The test cases start from the next page.

\includepdf[pages=-]{../FunctionInfo/test-inputs/others.pdf}

\textbf{Expected Results}:\\

\begin{verbatim}
Module test-inputs/others.bc
Name,   Args,   Calls,  Blocks, Insns
z,      0,      0,      1,      1
y,      1,      1,      3,      6
loop,   3,      1,      0,      0
FindMax,        *,      0,      8,      29
llvm.va_start,  1,      1,      0,      0
llvm.va_end,    1,      1,      0,      0
\end{verbatim}

%%% Local Variables:
%%% mode: latex
%%% TeX-master: "asst1"
%%% End:
