\documentclass[letterpaper]{article}

\usepackage{hw}
\usepackage{bm}
\usepackage{amsmath}
\usepackage{graphicx}
\usepackage[colorlinks=false,urlcolor=blue]{hyperref}
\usepackage{geometry}
\geometry{margin=1in}
\usepackage{multicol}
\usepackage{paralist}
\usepackage{todonotes}
\setlength{\marginparwidth}{2.15cm}
\usepackage{booktabs}
\usepackage{enumitem}
\usepackage{cleveref}
\usepackage{pdfpages}
\usepackage{fancyhdr}
\usepackage{verbatim}
\usepackage{tikz}
\usetikzlibrary{arrows}

%\DeclareMathOperator*{\argmin}{arg\,min}
%\DeclareMathOperator*{\argmax}{arg\,max}

\newcommand{\email}[1]{\href{mailto:#1@cs.cmu.edu}{#1}}

\pagestyle{fancy}
\lhead{AID :: \email{jarulraj}, \email{akalia}, \email{junwoop}}

\begin{document}

\section*{}
\begin{center}
  \centerline{\textbf{\Large Optimizing Matrix Operations Using Novel DRAM
  Access Primitives}}
  \vspace{1em}
  \textsc{\large CMU 15-745: Optimizing Compilers (Spring 2015)} \\
  \vspace{3em}
  \centerline{\large{\textbf{Group} : Joy Arulraj (\email{jarulraj}), Anuj Kalia
  (\email{akalia})}, Jun Woo Park (\email{junwoop}) }
  \vspace{1em}
\end{center}

\section{Major Changes}

There has been no major changes to the goals we set out in the initial report. We have not met any obstacles so far and we still believe that the implementation strategy is the best way to proceed. 

\section{Accomplishment So Far}

We have conducted preliminary evaluation of the potential performance impact of the transformation. Specifically, we measured the time to compute row sum (sum of all element within a row) and column sum (sum of all element within a column) for a matrix laid out in row major order. 

We implemented a Compiler pass to identify row order and column order access patterns for multidimensional arrays as well as for linearized arrays. We parse and use programmer annotation to determine array variable we need to analyze. Then for each loop within a function, we analyze every load and store operation using Scalar Evolution pass within LLVM framework to determine access strides and offsets. 

We are still implementing transforming access pattern to leverage the DRAM primitives. 

\section{Meeting Milestone}

\section{Surprises}

\section{Revised Schedule}

\begin{itemize}

\item \textbf{Plan of Attack and Schedule:}

\textbf{Weeks 5-6 :}

\begin{itemize}
\item Joy Arulraj : Implement a matrix library along with annotations of access
patterns and simulate the novel DRAM access primitives. \footnote{We plan to
simulate them by maintaining multiple versions of the matrix laid out in both row
and column-major orders. We will then redirect the software accesses to
the appropriate version.}

\item Anuj Kalia : Transform application access patterns to leverage the DRAM
primitives.

\item Jun Woo Park : Implement a mechanism to automatically identify application
access patterns.
\end{itemize}

\item \textbf{Milestone:} 

We plan to have a working compiler pass that transforms the matrix library 
with programmer annotations to leverage these DRAM access primitives.
We also hope to do some initial evaluation on the performance impact
of this transformation and present them in the milestone report.

\end{itemize}

\bibliographystyle{acm}
\bibliography{ref}

\end{document}

%%% Local Variables:
%%% mode: latex
%%% TeX-master: "."
%%% End:
