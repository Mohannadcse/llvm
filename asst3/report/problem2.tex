\section{Instruction Scheduling}

\begin{itemize}

\item \textbf{Constraints} \\
  ADD : 1 cycle \\
  SUB : 1 cycle \\
  MUL : 4 cycles \\
  DIV : 7 cycles \\
  LD  : 3 cycles \\
  ST  : 1 cycle \\

  Processor : 2 instructions/cycle \\
  Functional Unit : 1 instruction/cycle \\

\item \textbf{Instructions}

  The list of instructions to be scheduled along with their dependencies are shown below.
  
\begin{table}
\centering
\begin{tabular}{c|l|l|l}
  \toprule
  \toprule
  \textbf{Id} & \textbf{Instruction} & \textbf{Cycles} & \textbf{Depends} \\
  \midrule
  I-1   &  LD a         & 1  &          \\ \hline
  I-2   &  LD b         & 1  &          \\ \hline
  I-3   &  ADD c, a, b  & 1  & I-1, I-2 \\ \hline
  I-4   &  LD d         & 1  &          \\ \hline
  I-5   &  SUB e, b, a  & 1  & I-1, I-2 \\ \hline
  I-6   &  DIV f, a, d  & 7  & I-1, I-4 \\ \hline
  I-7   &  ST e         & 1  & I-5      \\ \hline
  I-8   &  LD g         & 1  &          \\ \hline
  I-9   &  MUL h, e, b  & 4  & I-2, I-5 \\ \hline
  I-10  &  MUL i, d, d  & 4  & I-4      \\ \hline
  I-11  &  ADD j, g, i  & 1  & I-8, I-10 \\ \hline
  I-12  &  ST c         & 1  & I-3      \\ \hline
  I-13  &  ST h         & 1  & I-9      \\ \hline
  I-14  &  ST j         & 1  & I-11     \\ \hline
  I-15  &  ST f         & 1  & I-6      \\ \hline
  \bottomrule
\end{tabular}
\end{table}  
  
\item \textbf{Forward list scheduling}

  In this case, the edges in the data precedence graph are in forward direction.
  We break ties by selecting the instruction that appeared first in the regular
  program order.
  
\begin{table}
\centering
\begin{tabular}{c|l|l|l|l}
  \toprule
  \toprule
  \textbf{Cycle} & \textbf{ALU-1} & \textbf{ALU-2} & \textbf{MEM} & \textbf{READY} \\
  \midrule
  0  &      &      &  I-4  & I-4, I-1, I-2, I-8 \\ \hline
  1  &      &      &  I-1  & I-1, I-2, I-8      \\ \hline
  2  &      &      &  I-2  & I-2, I-8           \\ \hline
  3  &      & I-10 &  I-8  & I-8, I-10          \\ \hline
  4  & I-6  &      &       & I-6                \\ \hline
  5  & I-5  & I-3  &       & I-5, I-3           \\ \hline
  6  & I-9  &      &  I-7  & I-9, I-7, I-12     \\ \hline
  7  & I-11 &      &  I-12 & I-11, I-12         \\ \hline
  8  &      &      &  I-14 & I-14               \\ \hline
  9  &      &      &       &                    \\ \hline
  10 &      &      &  I-13 & I-13               \\ \hline
  11 &      &      &  I-15 & I-15               \\ \hline
  \bottomrule
\end{tabular}
\caption{Forward list scheduling algorithm's output.}
\end{table}

\item \textbf{Backward list scheduling}

  In this case, the edges in the data precedence graph are in backward direction.
  We break ties by selecting the instruction that appeared first in the regular
  program order. We schedule the finish times of each operation and are constrained
  by FU availability.

  We observe that while it takes 12 cycles in the previous case, it only takes 11
  cycles while using this scheduling algorithm.
  
\begin{table}
\centering
\begin{tabular}{c|l|l|l|l}
  \toprule
  \toprule
  \textbf{Cycle} & \textbf{ALU-1} & \textbf{ALU-2} & \textbf{MEM} & \textbf{READY} \\
  \midrule
  0  &      &      &  I-4  &                               \\ \hline
  1  &      &      &  I-2  &                               \\ \hline
  2  &      &      &  I-1  & I-4                           \\ \hline
  3  & I-6  & I-10 &       & I-2                           \\ \hline
  4  & I-5  &      &  I-8  & I-5, I-1, I-8                 \\ \hline
  5  & I-9  & I-3  &       & I-3, I-8                      \\ \hline
  6  &      &      &  I-12 & I-10, I-12, I-8               \\ \hline
  7  &      & I-11 &  I-7  & I-11, I-7, I-12               \\ \hline
  8  &      &      &  I-14 & I-9, I-14, I-7, I-12          \\ \hline
  9  &      &      &  I-13 & I-6, I-13, I-14, I-7, I-12    \\ \hline
  10 &      &      &  I-15 & I-15, I-13, I-14, I-7, I-12   \\ \hline
  \bottomrule
\end{tabular}
\caption{Backward list scheduling algorithm's output.}
\end{table}

\end{itemize}

